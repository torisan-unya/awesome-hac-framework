\documentclass[a4paper]{article}

%=========== Preamble ===========
\usepackage[utf8]{inputenc}
\usepackage[T1]{fontenc}
\usepackage{amsmath}
\usepackage{amssymb}
\usepackage{geometry}
\geometry{a4paper, margin=1in}
\usepackage{graphicx}
\usepackage[
    colorlinks=true,
    linkcolor=blue,
    urlcolor=cyan,
    citecolor=green
]{hyperref}
\usepackage{url}
\urlstyle{same} % Use the same font as the text for URLs

% For tables
\usepackage{longtable}
\usepackage{booktabs}
\usepackage{array}
\usepackage{calc} % For \real in table column width calculation

% For code highlighting
\usepackage{minted}

% For special symbols
\usepackage{textcomp}

\begin{document}

%=========== Document Body ===========

\section{Enhancing Explainability in Healthcare AI through the Extended
Collaborative Intelligence Index (X-CII): A Synthetic Evaluation
Framework}\label{enhancing-explainability-in-healthcare-ai-through-the-extended-collaborative-intelligence-index-x-cii-a-synthetic-evaluation-framework}

\textbf{Author:} Torisan Unya (ORCID: \url{https://orcid.org/0009-0004-7067-9765})\\
\textbf{Affiliation:} Independent Researcher\\
\textbf{Keywords:} Human-AI Collaboration, Collaborative Intelligence
Metrics, Explainable AI, Healthcare Applications, X-CII Framework,
Synthetic Evaluation, Regulatory Compliance\\
\textbf{Categories:} cs.HC; cs.AI; cs.LG; stat.ML\\
\textbf{Submission Note:} v1: Added dedicated Methods and Results
sections for enhanced structure and clarity; recomputed statistics using
verified code execution (e.g., baseline median updated to 102.963\% to
reflect precise floating-point results); incorporated recent XAI
advancements from semantic search; confirmed EU AI Act details with
official sources. Verified references as of September 29, 2025. No
empirical claims; synthetic evaluation only. Reproducible code provided
in Appendix A (requires Python 3.10+, NumPy 1.23+, SciPy 1.10+;
execution time approximately 0.5 seconds on standard hardware). Licensed
under CC BY-SA 4.0 (paper) and MIT (code).

\subsection{Plain-Language Summary for
Clinicians}\label{plain-language-summary-for-clinicians}

In healthcare, AI tools can help with diagnosis, but their ``black box''
nature creates risks. If an AI suggests a treatment without a clear
reason, how can you confidently explain it to a patient or justify it
legally?\\
X-CII is a ``report card'' for a human-AI team. It grades not just
accuracy ($Q$) or speed ($E$), but also how well the team works together to
make safe, explainable decisions ($S$). A good AI explanation acts like
better teamwork, boosting the team's diagnostic ``detection skill'' by a
simulated 5\%. Our simulations showed this led to a typical performance
improvement of around 3\% for the human-AI team compared to the best
individual (human or AI) working alone. In fact, the collaborative team
was the better option in around 90\% of cases.\\
This approach aligns with new regulations like the EU AI Act\textsuperscript{1} and IMDRF
guidelines, which require transparency for high-risk medical AI. By
measuring the value of an explanation, X-CII helps build trustworthy AI
that supports clinicians and meets legal standards. This is a conceptual
framework, not for direct clinical use, and should be calibrated to
site-specific data in practice.\\
\textsuperscript{1}Obligations under these regulations are being introduced in stages over
the next few years. Check the latest version for updates.

\subsection{Abstract}\label{abstract}

Human-AI collaboration in healthcare demands explainable AI (XAI) to
foster trust, safety, and regulatory compliance, as mandated by the EU
AI Act {[}1{]} and IMDRF Good Machine Learning Practice (GMLP) {[}2{]}.
This paper formalizes the Extended Collaborative Intelligence Index
(X-CII) to quantify XAI's role in enhancing collaborative performance.
X-CII aggregates quality ($Q$), efficiency ($E$), and safety ($S$) via a power
mean ($\lambda=0.25$), isolating explainability's impact as a +5\% uplift to the
team's detectability index ($d'$ in Signal Detection Theory, assuming
equal-variance Gaussian). In this synthetic study, this uplift is
applied only to collaborative $d'$ and reflects explainability's effect on
Safety; $Q$/$E$ are fixed in baseline to isolate mechanisms. This
conservative estimate is derived from literature reporting 5-10\%
performance gains from XAI integration (ranges vary by task and study
quality)\textsuperscript{1}; for example, an AUC of 0.800 ($d'\approx1.189$) with +5\% uplift
yields $d'\approx1.248$ and AUC$\approx0.813$ (+1.3 percentage points). Addressing
critiques of post-hoc XAI (e.g., Rudin, 2019 {[}3{]}), we emphasize
X-CII's support for fidelity in explanations, mitigating risks like
inappropriate reliance and bias amplification via explicit factors in
$S$.\\
Synthetic Monte Carlo simulations (10,000 replicates) illustrate
relative X-CII around 102.963\% median (IQR: 101.236--104.560\%) vs.~the
better baseline, contingent on baseline human/AI skill, prevalence, and
cost ratios. Revised Safety normalization ($1 - L / L_{\text{worst}}$) preserves
boundedness and comparability; while it ensures consistency, it
compresses differences in high-performance regimes. Sensitivity analyses
include uplift on team-only (median 102.963\%), single-only (102.130\%),
both (102.959\%); $\lambda$ variations (geometric mean: 103.051\%, arithmetic:
102.694\%); $\eta$ (0.6: 95.055\%, 0.8: 99.156\%, 1.0: 102.963\%); $\rho$ (-0.5:
108.659\%, 0.5: 99.637\%); and domain shifts with mild single-agent F/R
adjustments (AUC=0.72: median 102.818\%, win rate 78.5\%). Under domain
shifts (AUC=0.72, applied uniformly), medians $\sim$102.818\%
(win rate $\sim$78.5\%). Baseline win rate (relative X-CII
$>$100\%) is approximately 89.7\% under independent assumptions
($\rho=0$). Results are illustrative, setting-dependent, and should not be
generalized without empirical validation. We integrate multimodal
foundation models (MFMs) and generative AI challenges conceptually,
using uncertainty quantification (e.g., semantic entropy, with reported
AUCs of $\sim$0.75-0.85 across datasets as representative
examples {[}6{]}) to address confabulation risks (conceptually
referenced, not implemented in code). This framework supports
implementation toward compliance in human-AI teams, but does not itself
establish conformity.\\
\textsuperscript{1}For example, systematic reviews report 5--10\% improvements in
diagnostic tasks {[}16{]}, and experimental studies of human--AI
interaction similarly show performance gains in the 5--10\% range
{[}17{]}.

\subsection{Introduction}\label{introduction}

AI integration in healthcare enhances diagnostics and treatment but
faces opacity challenges, risking trust erosion and non-compliance with
regulations like the EU AI Act (Reg. (EU) 2024/1689; published in
Official Journal L 206 on July 12, 2024; entered into force on August 1,
2024; key obligations phased: prohibitions February 2, 2025; General
Purpose AI (GPAI) August 2, 2025; high-risk August 2, 2027) {[}1{]} and
IMDRF GMLP {[}2{]}.\textsuperscript{1} XAI mitigates this by providing interpretable
insights, enabling clinicians to understand AI decisions and justify
them to patients or regulators. This paper extends the Collaborative
Intelligence Index (CII) to X-CII, incorporating XAI's impact on team
performance via Signal Detection Theory (SDT) metrics {[}13--15, 18{]}.
We focus on synthetic evaluation to isolate mechanisms, assuming
equal-variance Gaussian noise and conservative +5\% uplift to
collaborative $d'$ from XAI, derived from literature reviews and studies
reporting 5--10\% gains in task performance {[}16,17{]}. This uplift
models improved detectability through better fidelity and calibrated
reliance, reducing errors like over-reliance on flawed explanations
{[}3,17{]}.\\
The framework addresses key XAI challenges: post-hoc vs.~intrinsic
interpretability {[}3{]}, multimodal integration, and uncertainty
quantification {[}6,7{]}. X-CII quantifies these through $S$
(incorporating fidelity $F$ and reliance $R$), while maintaining fixed $Q$/$E$
in baselines. We demonstrate via synthetic Monte Carlo that XAI-induced
uplifts yield consistent collaborative advantages, though domain shifts
compress benefits. This work complements recent XAI reviews {[}8--11{]}
and SDT applications {[}13{]}, providing a reproducible metric for
high-risk AI compliance. Limitations include synthetic assumptions;
future work should adapt and calibrate to real datasets.\\
\textsuperscript{1}Specific obligations are phased in over 36 months; prohibitions after 6
months, GPAI after 12, and high-risk systems after 24-36 months. See
CELEX: 32024R1689 for details.

\subsection{Methods}\label{methods}

\subsubsection{Synthetic Simulation
Setup}\label{synthetic-simulation-setup}

We employ Monte Carlo simulations (10,000 replicates) to evaluate X-CII
robustness under parameterized uncertainty. Parameters are drawn from
literature-informed priors: AUC $\sim$ Uniform(0.75,0.85)
(converted to $d'$ via inverse normal: $d' = \sqrt{2} \cdot \Phi^{-1}(\text{AUC})$); $\pi$ (prevalence)
$\sim$ Beta(6,14) for moderate imbalance; $c_{\text{FN}}$
$\sim$ Uniform(2,5), $c_{\text{FP}}$ $\sim$ Uniform(0.5,2)
for asymmetric costs {[}13,18{]}. $Q$ and $E$ are fixed at 0.75 in baselines
to isolate $S$ effects. Based on prior studies {[}16,17{]}, uplift is
assumed at +5\%, though this varies by task; $\rho=0$ baseline (sensitivity
$\pm0.5$), $\eta=1.0$ baseline (sensitivity 0.6-0.8). These assumptions aim to
illustrate model behavior under plausible parameters.\\
Team $d'$ is calculated using a Mahalanobis distance generalization; in
our implementation this reduces to the correlated Gaussian formulation
(see Appendix A for code; $\rho=0$ baseline, sensitivity $\pm0.5$; $\eta=1.0$
baseline, sensitivity 0.6-0.8). XAI uplift (+5\%) is applied
multiplicatively to $d_{\text{team}}$ (team-only baseline) or variants
(single-only, both).\\
Expected loss $L$ is minimized under SDT: $L = c_{\text{FN}} \cdot \pi \cdot (1 - \text{TPR}) + c_{\text{FP}} \cdot (1 - \pi) \cdot \text{FPR}$, with
optimal threshold $\tau^* = 0.5 d' + \log((c_{\text{FP}}(1-\pi))/(c_{\text{FN}} \pi)) / d'$
(equal-variance Gaussian). Safety $S = 1 - L / L_{\text{worst}}$, scaled by $(\alpha +
(1-\alpha)F)(1-R)$ ($\alpha=0.5$; $F=1.0$ baseline human/AI, 0.95 collab; $R=0.0$
baseline human/AI, 0.05 collab; mild adjustments under shifts).\\
X-CII = $((Q^{\lambda} + E^{\lambda} + S^{\lambda})/3)^{1/\lambda}$ ($\lambda=0.25$; geometric
$\lambda\to0$, arithmetic $\lambda=1$ sensitivities). Relative X-CII = $100 \cdot \text{X-CII}_{\text{collab}}
/ \max(\text{X-CII}_{\text{human}}, \text{X-CII}_{\text{ai}})$. Domain shifts: AUC=0.72 fixed, F/R
adjusted (human/AI: $F=0.98$, $R=0.02$; collab: $F=0.92$, $R=0.08$). Seed=42 for
reproducibility.

\subsubsection{Statistical Analysis}\label{statistical-analysis}

Medians/IQRs via NumPy percentiles; win rates as proportion relative
$>$100\%. Sensitivities vary one parameter at a time. No
hypothesis testing; illustrative only.

\subsection{Results}\label{results}

Synthetic simulations demonstrate consistent collaborative advantages.
Baseline relative X-CII: median 102.963\% (IQR: 101.236--104.560\%), win
rate 89.7\%. Table 1 summarizes sensitivities.

\begin{longtable}[]{@{}
  >{\raggedright\arraybackslash}p{(\linewidth - 6\tabcolsep) * \real{0.3140}}
  >{\raggedright\arraybackslash}p{(\linewidth - 6\tabcolsep) * \real{0.3140}}
  >{\raggedright\arraybackslash}p{(\linewidth - 6\tabcolsep) * \real{0.2093}}
  >{\raggedright\arraybackslash}p{(\linewidth - 6\tabcolsep) * \real{0.1628}}@{}}
\toprule
\begin{minipage}[b]{\linewidth}\raggedright
Scenario
\end{minipage} & \begin{minipage}[b]{\linewidth}\raggedright
Median Relative X-CII (\%)
\end{minipage} & \begin{minipage}[b]{\linewidth}\raggedright
IQR (\%)
\end{minipage} & \begin{minipage}[b]{\linewidth}\raggedright
Win Rate (\%)
\end{minipage} \\
\midrule
\endhead
\bottomrule
\endlastfoot
Baseline (team uplift) & 102.963 & 101.236--104.560 & 89.7 \\
Uplift single-only & 102.130 & 100.682--103.492 & 85.4 \\
Uplift both & 102.959 & 101.302--104.510 & 90.5 \\
$\lambda=0$ (geometric) & 103.051 & 101.324--104.648 & 89.7 \\
$\lambda=1$ (arithmetic) & 102.694 & 100.967--104.291 & 89.7 \\
$\eta=0.6$ & 95.055 & 94.093--95.951 & 0.0 \\
$\eta=0.8$ & 99.156 & 98.439--99.980 & 24.6 \\
$\rho=-0.5$ & 108.659 & 106.933--110.481 & 99.2 \\
$\rho=0.5$ & 99.637 & 98.892--100.466 & 37.9 \\
Shift AUC=0.72 & 102.818 & 101.091--104.415 & 78.5 \\
\end{longtable}

\emph{Table 1: Sensitivity analyses of relative X-CII. Win Rate = \% of
simulations where collaborative $>$ max(human, AI). All
values from 10,000 replicates; illustrative only.}

Under shifts, benefits compress but remain positive. Results align with
literature {[}16,17{]}, though synthetic.

\subsection{Discussion}\label{discussion}

X-CII formalizes XAI's value in healthcare human-AI teams, showing
modest but consistent uplifts via synthetic evaluation. Integration with
MFMs {[}7{]} and uncertainty metrics {[}6{]} enhances conceptual
robustness. The high sensitivity of X-CII to $\eta$ (team efficiency) and $\rho$
(skill correlation) quantitatively highlights the importance of not only
technical performance but also team composition and training in AI
implementation. This study is a simulation to verify the theoretical
validity and sensitivity of the X-CII framework. The presented values
(e.g., collaborative superiority in 90\% of cases) indicate potential
under the assumed parameters and do not directly predict real clinical
outcomes. Future research requires calibration and validation using
actual data. The introduction of X-CII should be accompanied by careful
ethical and organizational considerations, including redefining
clinician roles, responsibilities, and addressing risks like deskilling
and automation bias. Limitations include synthetic assumptions (e.g.,
equal-variance, fixed $Q$/$E$); empirical validation is needed. Calibration
procedure is outlined in Appendix B. Future work: real-data adaptation,
interface factors, and regulatory pilots.

\subsection{Appendix A: Reproducible
Code}\label{appendix-a-reproducible-code}

\begin{minted}[
  fontsize=\small,
  frame=lines,
  framesep=2mm,
  breaklines,
  linenos
]{python}
import numpy as np
from scipy.stats import norm

def auc_to_dprime(auc):
    return np.sqrt(2) * norm.ppf(np.clip(auc, 1e-6, 1-1e-6))

def team_dprime(d_h, d_ai, rho=0.0, eta=1.0):
    rho = np.clip(rho, -0.999, 0.999)
    num = d_h**2 + d_ai**2 - 2 * rho * d_h * d_ai
    den = np.maximum(1 - rho**2, 1e-12)
    return eta * np.sqrt(np.maximum(num / den, 0.0))

def expected_loss(d_prime, pi, c_fp, c_fn):
    mu0 = 0.0
    mu1 = d_prime
    delta = np.maximum(mu1 - mu0, 1e-6)
    log_k = np.log(np.maximum(c_fp, 1e-12) * (1 - pi)) - np.log(np.maximum(c_fn, 1e-12) * np.maximum(pi, 1e-12))
    tau_star = 0.5 * (mu0 + mu1) + log_k / delta
    tpr = 1 - norm.cdf(tau_star - mu1)
    fpr = 1 - norm.cdf(tau_star - mu0)
    return c_fn * pi * (1 - tpr) + c_fp * (1 - pi) * fpr

def safety(L, pi, c_fp, c_fn, F, R, alpha=0.5):
    L_allow = (1 - pi) * c_fp
    L_block = pi * c_fn
    L_worst = np.maximum(L_allow, L_block)
    base = 1 - L / np.maximum(L_worst, 1e-6)
    return np.clip(base * (alpha + (1 - alpha) * F) * (1 - R), 0, 1)

def core_xcii(q, e, s, lam=0.25):
    q = np.clip(q, 1e-12, 1)
    e = np.clip(e, 1e-12, 1)
    s = np.clip(s, 1e-12, 1)
    if abs(lam) < 1e-12:
        return np.exp((np.log(q) + np.log(e) + np.log(s)) / 3)
    else:
        return ((q**lam + e**lam + s**lam) / 3)**(1 / lam)

def run_scenario(rho=0.0, eta=1.0, lam=0.25, uplift=1.05,
                 uplift_team=True, uplift_single=False,
                 auc_fixed=None,
                 F_h=1.0, R_h=0.0, F_ai=1.0, R_ai=0.0,
                 F_collab=0.95, R_collab=0.05,
                 seed=42, n=10000, include_stats=True):
    rng = np.random.default_rng(seed)
    auc_h = rng.uniform(0.75, 0.85, n) if auc_fixed is None else np.full(n, auc_fixed)
    auc_ai = rng.uniform(0.75, 0.85, n) if auc_fixed is None else np.full(n, auc_fixed)
    d_h = auc_to_dprime(auc_h)
    d_ai = auc_to_dprime(auc_ai)
    pi = rng.beta(6, 14, n)
    c_fn = rng.uniform(2, 5, n)
    c_fp = rng.uniform(0.5, 2, n)
    if uplift_single:
        d_h_eff = d_h * uplift
        d_ai_eff = d_ai * uplift
    else:
        d_h_eff, d_ai_eff = d_h, d_ai
    d_team = team_dprime(d_h_eff, d_ai_eff, rho=rho, eta=eta)
    if uplift_team:
        d_team = d_team * uplift
    L_h = expected_loss(d_h_eff, pi, c_fp, c_fn)
    L_ai = expected_loss(d_ai_eff, pi, c_fp, c_fn)
    L_team = expected_loss(d_team, pi, c_fp, c_fn)
    alpha = 0.5
    q = e = np.full(n, 0.75)
    s_h = safety(L_h, pi, c_fp, c_fn, F_h, R_h, alpha)
    s_ai = safety(L_ai, pi, c_fp, c_fn, F_ai, R_ai, alpha)
    s_c = safety(L_team, pi, c_fp, c_fn, F_collab, R_collab, alpha)
    core_h = core_xcii(q, e, s_h, lam)
    core_ai = core_xcii(q, e, s_ai, lam)
    core_c = core_xcii(q, e, s_c, lam)
    rel = 100 * core_c / np.maximum(core_h, core_ai)
    median = np.median(rel)
    iqr = np.percentile(rel, [25, 75])
    win_rate = (rel > 100).mean() * 100
    if include_stats:
        mean = np.mean(rel)
        std = np.std(rel)
        return median, iqr, win_rate, mean, std
    return median, iqr, win_rate

# Baseline and sensitivities
print("Baseline:", run_scenario())
print("λ=0 (geom):", run_scenario(lam=0))
print("λ=0.5:", run_scenario(lam=0.5))
print("λ=1.0 (arithmetic):", run_scenario(lam=1.0))
print("η=0.6:", run_scenario(eta=0.6))
print("η=0.8:", run_scenario(eta=0.8))
print("ρ=-0.5:", run_scenario(rho=-0.5))
print("ρ=0.5:", run_scenario(rho=0.5))
print("uplift single-only:", run_scenario(uplift_team=False, uplift_single=True))
print("uplift both:", run_scenario(uplift_team=True, uplift_single=True))
# Domain shift (AUC=0.72) + F/R adjustments
print("Shift AUC=0.72:", run_scenario(auc_fixed=0.72,
      F_h=0.98, R_h=0.02, F_ai=0.98, R_ai=0.02, F_collab=0.92, R_collab=0.08))
\end{minted}

\subsection{Appendix B: Calibration Checklist and
Alternatives}\label{appendix-b-calibration-checklist-and-alternatives}

This appendix outlines proposed methodology for calibrating X-CII in
real environments.\\
- $\rho$ estimation: Compute Pearson/Spearman on z-transformed paired scores;
stratify by case-mix; robustify with bootstrapping and binormal ROC
fitting for covariance; pool within-class covariances (estimate $\rho_0$, $\rho_1$
and variances per class 0/1, weight by sample size under
equal-variance); check equal-variance via slope $s \approx1$. Use split
cross-validation to avoid leakage in experiments collecting both human
confidence scores and AI outputs on the same cases.\\
- F/R calibration: Counterfactual surveys, calibration curves. For $\alpha$:
SOP compliance rate (e.g., documentation adherence threshold 0.8,
double-check rates $>$0.9); $F$: explanation fidelity from blind
evaluations (threshold 0.9); $R$: over-/under-reliance from behavioral
metrics (threshold 0.1), normalized.\\
- Alternative Safety normalizations (not used; for illustration):
Log-compressed: $1 - \log(1 + L/L_{\text{ref}})/\log(1 + L_{\text{worst}}/L_{\text{ref}})$;
Power-transformed: $1 - (L/L_{\text{worst}})^{0.5}$ (expands high-performance
differences).

\subsection{References}\label{references}

{[}1{]} European Parliament and Council. (2024). Regulation (EU)
2024/1689\ldots{} Official Journal of the European Union, L 206,
12.7.2024, p.~1--252. CELEX: 32024R1689.\\
{[}2{]} IMDRF. (2025). Good Machine Learning Practice\ldots{} IMDRF/AIML
WG/N88 FINAL:2025. Published January 29, 2025.\\
{[}3{]} Rudin, C. (2019). Stop explaining black box\ldots{} Nature
Machine Intelligence, 1(5), 206-215. DOI: 10.1038/s42256-019-0048-x.\\
{[}4{]} Fragiadakis, G., et al.~(2024). Evaluating Human-AI
Collaboration\ldots{} arXiv:2407.19098 {[}cs.HC{]}.\\
{[}5{]} Hildt, E. (2025). What Is the Role of Explainability\ldots{}
Frontiers in Digital Health, 7. DOI: 10.3389/fdgth.2025.12025101.\\
{[}6{]} Farquhar, S., et al.~(2024). Detecting hallucinations\ldots{}
Nature, 630(8017), 625-630. DOI: 10.1038/s41586-024-07421-0.\\
{[}7{]} Chen, R. J., et al.~(2024). Towards multimodal foundation
models\ldots{} arXiv:2402.09849 {[}cs.LG{]}.\\
{[}8{]} Giorgetti, C., et al.~(2025). Healthcare AI,
explainability\ldots{} Frontiers in Medicine, 12:1545409. DOI:
10.3389/fmed.2025.1545409.\\
{[}9{]} Mohapatra, R. K. (2025). Advancing explainable AI\ldots{}
Computers in Biology and Medicine, 119:108599. DOI:
10.1016/j.compbiomed.2025.108599.\\
{[}10{]} El-Geneedy, M., et al.~(2025). A comprehensive explainable
AI\ldots{} Scientific Reports, 15(1):11263. DOI:
10.1038/s41598-025-11263-9.\\
{[}11{]} Vani, M. S., et al.~(2025). Personalized health
monitoring\ldots{} Scientific Reports, 15(1):15867. DOI:
10.1038/s41598-025-15867-z.\\
{[}12{]} CADTH. (2025). 2025 Watch List\ldots{} NCBI Bookshelf ID:
NBK613808.\\
{[}13{]} Kovesdi, C., et al.~(2025). Application of Signal Detection
Theory\ldots{} Proceedings of the Human Factors and Ergonomics Society
Annual Meeting. DOI: 10.1177/10711813251368829.\\
{[}14{]} Green, D. M., \& Swets, J. A. (1966). Signal Detection Theory
and Psychophysics. Wiley.\\
{[}15{]} Macmillan, N. A., \& Creelman, C. D. (2005). Detection Theory:
A User's Guide (2nd ed.). Lawrence Erlbaum Associates. DOI:
10.4324/9781410611147.\\
{[}16{]} O'Connor, M., et al.~(2024). A systematic review\ldots{}
Computational and Structural Biotechnology Journal, 23:101-120. DOI:
10.1016/j.csbj.2024.07.015.\\
{[}17{]} Köhler, S., et al.~(2025). Interacting with fallible AI\ldots{}
Frontiers in Psychology, 16:1574809. DOI: 10.3389/fpsyg.2025.1574809.\\
{[}18{]} Sorkin, R. D., \& Dai, H. (1994). Signal detection analysis of
the ideal group. Organizational Behavior and Human Decision Processes,
60(1), 1-13. DOI: 10.1006/obhd.1994.1072.

\end{document}
```
